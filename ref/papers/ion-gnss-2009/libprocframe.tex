\section*{Additional Library Enhancements}

In contrast to the RINEX version 3.00 development, smaller scale
enhancements have been conducted in the trunk. The
\gpstkapp{procframe} library, which provide precise point positioning
capabilities, has been upgraded. Also, a new library has been
developed to support vector-based graphical output such as plots.

\subsection*{Processing Framework}

The GPSTk Processing Framework is based on the ``GNSS Data Structures'' (GDS)
and the associated ``GDS Processing Paradigm'' first presented at
\cite{ion:gnss07}. Its purpose is to solve data management problems that are
difficult to solve with simpler structures like vectors and matrices.

Moreover, the ``GDS Processing Paradigm'' provides an unified and consistent way
to handle such structures, providing {\it processing classes} that handle both
data and corresponding {\it metadata}. The objects from these {\it processing
classes} reach into the GDS and add, delete and/or modify what is needed
(according to their function), and leave the results in the same GDS,
appropriately indexed. These processing objects are designed to use sensible
defaults in their parameters, but may be tuned to suit specific needs.

The former ideas are coupled with a handy redefinition of C++ operator
\gpstkcommand{>>}, implemented in such a way that several operators may be
concatenated. This allows a programming style that clearly shows how the data is
{\it flowing} along the processing steps, in a way that is similar to Unix
``pipes''. A simple example of this approach is presented. Here, just a single
epoch worth of data will be extracted out of a RINEX observation file, and put
into a GDS:

\begin{scriptsize}
\begin{lstlisting}
1   RinexObsStream rinexFile("ebre0300.02o");
2   gnssRinex gpsData;

3   rinexFile >> gpsData;
\end{lstlisting}
\end{scriptsize}


Line \#1 declares an object named \gpstkobj{rinexFile} of class
\gpstkclass{RinexObsStream}, which is used to handle RINEX observation files.
That object will take care of handling ``ebre0300.02o'' RINEX observation file.
Line \#2 declares an object of class \gpstkclass{gnssRinex}, which is a very
common and handy GDS (several structures are available). Data will be
stored in this object, called \gpstkobj{gpsData}.

Then, line \#3 does the real work: It will take one epoch of data out of
\gpstkobj{rinexFile} and will pour it into \gpstkobj{gpsData}. No further coding is needed
for this action, and line \# 3 is referred to as the ``processing line''.

Taking this approach to GNSS data management and processing as a starting point,
the capabilities of the Processing Framework have been greatly expanded during
last year. Several sophisticated processing classes have been added, and a
``Precise Point Positioning'' (PPP) implementation \cite{ppp:kouba2001} has been
fully developed.

PPP is a complex task, and issues like wind-up effects, solid, oceanic and polar
tides, and antenna phase centers variations, must be taken into account. Also,
the data rates of typical IGS products being used often do not match with data
rates from available observations, and therefore some time management issues
also arise. In order to deal with these tasks, the GPSTk now provides some
accessory classes like the following:

\gpstkclass{ConfDataReader}: Powerful class to parse and manage configuration
files. It supports multiple sections, variable descriptions and value
descriptions (such as units), and a wide range of variable types.

\gpstkclass{AntexReader} and \gpstkclass{Antenna}: These classes allow one to
properly use antenna phase center variations, both in ``relative'' as well as in
``absolute'' modes, reading them from IGS ANTEX format files. These classes
are complemented with processing classes that will take care of applying the
corresponding corrections: \gpstkclass{CorrectObservables} to manage receiver
antenna corrections, and \gpstkclass{ComputeSatPCenter} to handle satellite
antenna corrections.

\gpstkclass{SolidTides}, \gpstkclass{OceanLoading} and \gpstkclass{PoleTides}:
Compute tidal effects due to solid tides, ocean loading and pole
displacements, respectively.

For decimation, the \gpstkclass{Decimate} class is used. Given the usual data rate
mismatch between IGS precise products (900 s per sample) and RINEX observations
(30 s per sample) this is a recommended procedure. \gpstkclass{Decimate} class
takes advantage of GPSTk's sophisticated exception handling mechanism, and if data
epoch is not a multiple of 900 seconds (or other preset time interval) then the
\gpstkclass{Decimate} object will issue an ``exception'' (effectively halting
further processing of data for that epoch) and the program will continue
processing the next epoch. Decimation is thence achieved in a effective and
compact way.

A typical PPP processing line looks like follows:


\begin{scriptsize}
\begin{lstlisting}
gpsData >> requireObs >> linear1 >> markCSLI >> markCSMW
        >> markArc >> decimateData >> basicModel
        >> eclipsedSV >> grDelay >> svPcenter >> corr
        >> wUp >> computeTropo >> linear2 >> pcFilter
        >> phaseAlign >> linear3 >> baseChange >> cDOP
        >> pppSolver;
\end{lstlisting}
\end{scriptsize}


This GDS processing data chain is a single C++ line, although for the
sake of clarity it spans several physical lines. This line must be enclosed within a
\gpstkobj{while} loop to process all available epochs, and also within a
\gpstkobj{try-catch} block to manage exceptions. Further details are provide
by examples in the source distribution as ``example8.cpp'', ``example9.cpp'' and
``example10.cpp''. Detailed information is also availble in the Doxygen documentation provided on the website. The Doxygen documentation can also be generated from the source code as described in prior sections.

The following is a brief explanation about what some of these objects do,
and what classes they belong to:

\begin{itemize}

\item \gpstkobj{requireObs} (an object of class \gpstkclass{RequireObservables}): Checks if required
{\it TypeID}s (usually observations) are present.

\item \gpstkobj{linear1}, \gpstkobj{linear2} and \gpstkobj{linear3}
(\gpstkclass{ComputeLinear}): They compute linear combinations such as
ionospheric, Melbourne-Wubbena, ionosphere-free, etc.

\item \gpstkobj{markCSLI} (\gpstkclass{LICSDetector2}) and \gpstkobj{markCSMW}
(\gpstkclass{MWCSDetector}): They detect and mark cycle slips using the
ionospheric and Melbourne-Wubbena combinations, respectively. Besides,
{\it markArc} (\gpstkclass{SatArcMarker}) keeps track of satellite arcs.

\item \gpstkobj{decimateData} (\gpstkclass{Decimate}): Decimates data if epoch is not
a multiple of 900~s.

\item \gpstkobj{basicModel} (\gpstkclass{BasicModel}): Computes the basic components
of a GNSS signal propagation model.

\item \gpstkobj{eclipsedSV} (\gpstkclass{EclipsedSatFilter}): Removes from the GDS
the satellites in eclipse.

\item \gpstkobj{grDelay} (\gpstkclass{GravitationalDelay}): Computes gravitational
delay effect due to changing gravity field along satellite-receiver ray.

\item \gpstkobj{svPcenter} (\gpstkclass{ComputeSatPCenter}): Computes the effect of
satellite antenna phase center.

\item \gpstkobj{corr} (\gpstkclass{CorrectObservables}): Corrects observables from
tides, receiver antenna phase center, eccentricity, etc.

\item \gpstkobj{wUp} (\gpstkclass{ComputeWindUp}): Computes phase wind-up.

\item \gpstkobj{computeTropo} (\gpstkclass{ComputeTropModel}): Models delays due to
tropospheric effects.

\item \gpstkobj{pcFilter} (\gpstkclass{SimpleFilter}): Filters out spurious values in
the PC (ionosphere-free) combination.

\item \gpstkobj{phaseAlign} (\gpstkclass{PhaseCodeAlignment}): Aligns phase with code
values, preserving ambiguities integer nature.

\item \gpstkobj{baseChange} (\gpstkclass{XYZ2NEU}): Prepares GDS to use a
North-East-UP reference frame in object \gpstkobj{pppSolver}.

\item \gpstkobj{cDOP} (\gpstkclass{ComputeDOP}): Computes DOP values.

\item \gpstkobj{pppSolver} (\gpstkclass{SolverPPP}): Solves the equation system using
an Extended Kalman filter (EKF).
\end{itemize}

Particular mention deserves object \gpstkobj{pppSolver}. This object is preconfigured
to solve the PPP equation system in a way consistent with~\cite{ppp:kouba2001}:
Coordinates are treated as constants (static), receiver clock is considered
white noise, and residual vertical wet tropospheric effect is processed with a
random walk stochastic model.

\gpstkclass{SolverPPP} objects are very easy to configure. For instance, if the
solution is needed in a North-East-Up reference frame (instead of the default
ECEF), we just need to insert a \gpstkclass{XYZ2NEU} in the processing chain
(like \gpstkobj{baseChange} above) and configure the solver object:


\begin{scriptsize}
\begin{lstlisting}
pppSolver.setNEU( true );
\end{lstlisting}
\end{scriptsize}


In order to carry out kinematic instead of static positioning, the coordinates
stochastic model being used is changed. For example, we consider coordinates as
white noise with sigma equal to 100.0 meters:


\begin{scriptsize}
\begin{lstlisting}
WhiteNoiseModel newCoordinatesModel( 100.0 );
pppSolver.setCoordinatesModel( &newCoordinatesModel );
\end{lstlisting}
\end{scriptsize}


Moreover, it is very easy to set different stochastic models for each dimension.
For instance, a surface vehicle could be modeled with restrictions in the
vertical coordinate, like this:

\begin{scriptsize}
\begin{lstlisting}
WhiteNoiseModel horizontalModel( 100.0 );
RandomWalkModel verticalModel( 0.01 );

pppSolver.setNEU( true );
pppSolver.setXCoordinatesModel( &horizontalModel );
pppSolver.setYCoordinatesModel( &horizontalModel );
pppSolver.setZCoordinatesModel( &verticalModel );
\end{lstlisting}
\end{scriptsize}


In summary, the Processing Framework approach has demonstrated being
very flexible and powerful, and results obtained with the current PPP
implementation are comparable with other GNSS data processing suites
(\cite{esa:navitec2008}, \cite{geomaticweek2009}). The current
development efforts within this framework aim to provide a uniform
data structure powered by classes able to carry out network-based data
processing, programmable on-the-fly generic solvers, and ambiguity
fixing capabilities.

