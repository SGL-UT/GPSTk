%\documentclass{article}
%\usepackage{fancyvrb}
%\usepackage{src/perltex}
%\usepackage{src/xcolor}
%\usepackage{listings}
%\usepackage{longtable}
%\usepackage{multirow}
%\RecustomVerbatimEnvironment{Verbatim}{Verbatim}{frame=single}
\definecolor{console}{rgb}{0.95,0.95,0.95}
\lstset{basicstyle=\ttfamily, columns=flexible, backgroundcolor=\color{console}}

\newcommand{\outputsize}{footnotesize}
\newcommand{\application}[1]{\emph{#1}}
\newcommand{\setconsole}{\lstset{basicstyle=\ttfamily, columns=flexible, backgroundcolor=\color{console}}}
\newcommand{\setfileio}{\lstset{basicstyle=\ttfamily, columns=flexible, backgroundcolor=\color{console}}}

\perlnewcommand{\getuse}[1]
{
        my $command = $_[0];
        $command = $command." > temp 2>&1 |head -n 15";
        system("$command");

	my $counter = 0;
	my $done = 0;
	my $output = "";
        open(input,"temp");

        while(my $line = <input>)
	{
		if($done == 0)
		{
			$output = $output.$line if $counter < 15;
			$output = $output." . . .\n" if $counter >= 15;
			$done = 1 if $counter >= 15;
		}

		$counter = $counter + 1;
	}

        close(input);

        return  "\\begin{\\outputsize}\n" . "\\begin{lstlisting}\n" .
                "> ".$_[0]."\n\n" . $output .
                "\\end{lstlisting}\n" . "\\end{\\outputsize}\n";
}

\perlnewcommand{\getrevision}[1]
{
	my $revision = $_[0];
	$revision =~ m/\$LastChangedRevision: ([^\$]*)/;
	return $1;
}

\perlnewcommand{\entry}[4]
{
	my $output = "$_[0] \& $_[1] \& \\multirow{$_[3]}{2.5in}{$_[2]} \\\\";

	my $cnt = 0;
	while($cnt < ($_[3]-1))
	{
		$output = $output." \& \& \\\\ ";
		$cnt = $cnt + 1;
	}
	
	return $output;
}


%\begin{document}

\index{RinSum!application writeup}

\section{\emph{RinSum}}
\subsection{Overview}
The application reads a RINEX file and summarizes it content.


\subsection{Usage}
\begin{\outputsize}
\begin{longtable}{lll}
\multicolumn{3}{c}{\application{RinSum}} \\
\multicolumn{3}{l}{\textbf{Optional Arguments}} \\
\entry{Short Arg.}{Long Arg.}{Description}{1}
\entry{-i}{--input}{Input file name(s)}{1}
\entry{-f}{}{file containing more options}{1}
\entry{-o}{--output}{Output file name}{1}
\entry{-p}{--path}{Path for input file(s)}{1}
\entry{-R}{--Replace}{Replace header with full one.}{1}
\entry{-s}{--sort}{Sort the PRN/Obs table on begin time.}{1}
\entry{-g}{--gps}{Print times in the PRN/Obs table as GPS times.}{2}
\entry{}{--EpochBeg}{Start time, arg is of the form YYYY,MM,DD,HH,Min,Sec}{2}
\entry{}{--GPSBeg}{Start time, arg is of the form GPSweek,GPSsow}{1}
\entry{}{--EpochEnd}{End time, arg is of the form YYYY,MM,DD,HH,Min,Sec}{2}
\entry{}{--GPSEnd}{End time, arg is of the form GPSweek,GPSsow}{1}
\entry{-h}{--help}{print syntax and quit.}{1}
\entry{-d}{--debug}{print debugging info.}{1}
\end{longtable}
\end{\outputsize}

\subsection{Examples}

\subsection{Notes}

%\end{document}
