%\documentclass{article}
%\usepackage{fancyvrb}
%\usepackage{src/perltex}
%\usepackage{src/xcolor}
%\usepackage{listings}
%\usepackage{multirow}
%\RecustomVerbatimEnvironment{Verbatim}{Verbatim}{frame=single}
\definecolor{console}{rgb}{0.95,0.95,0.95}
\lstset{basicstyle=\ttfamily, columns=flexible, backgroundcolor=\color{console}}

\newcommand{\outputsize}{footnotesize}
\newcommand{\application}[1]{\emph{#1}}
\newcommand{\setconsole}{\lstset{basicstyle=\ttfamily, columns=flexible, backgroundcolor=\color{console}}}
\newcommand{\setfileio}{\lstset{basicstyle=\ttfamily, columns=flexible, backgroundcolor=\color{console}}}

\perlnewcommand{\getuse}[1]
{
        my $command = $_[0];
        $command = $command." > temp 2>&1 |head -n 15";
        system("$command");

	my $counter = 0;
	my $done = 0;
	my $output = "";
        open(input,"temp");

        while(my $line = <input>)
	{
		if($done == 0)
		{
			$output = $output.$line if $counter < 15;
			$output = $output." . . .\n" if $counter >= 15;
			$done = 1 if $counter >= 15;
		}

		$counter = $counter + 1;
	}

        close(input);

        return  "\\begin{\\outputsize}\n" . "\\begin{lstlisting}\n" .
                "> ".$_[0]."\n\n" . $output .
                "\\end{lstlisting}\n" . "\\end{\\outputsize}\n";
}

\perlnewcommand{\getrevision}[1]
{
	my $revision = $_[0];
	$revision =~ m/\$LastChangedRevision: ([^\$]*)/;
	return $1;
}

\perlnewcommand{\entry}[4]
{
	my $output = "$_[0] \& $_[1] \& \\multirow{$_[3]}{2.5in}{$_[2]} \\\\";

	my $cnt = 0;
	while($cnt < ($_[3]-1))
	{
		$output = $output." \& \& \\\\ ";
		$cnt = $cnt + 1;
	}
	
	return $output;
}


%\begin{document}

\index{WhereSat!application writeup}
\section{\emph{WhereSat}}
\subsection{Overview}
This application uses input ephemeris to compute the predicted location of a 
satellite. The Earth-centered, Earth-fixed (ECEF) position of the satellite is 
reported. Optionally, the topocentric coordinates--azimuth, elevation, and 
range--can be generated. The user can specify the time interval between 
successive predictions. Also the output can generated in a format easily
imported into numerical packages.

\subsection{Usage}
\begin{\outputsize}
\begin{tabular}{lll}
\multicolumn{3}{l}{\textbf{Required Arguments}} \\
Short Arg. & Long Arg. & Description \\
-b & --broadcast=ARG & \multirow{2}{2.5in}{Specify a RINEX navigation file. The user may enter multiple files.} \\
& & \\
-p & --prn=NUM & Specify which SV to analuze. \\
& & \\
\multicolumn{3}{l}{\textbf{Optional Arguments}} \\
Short Arg. & Long Arg. & Description \\
-h & --help & Generates help and usage. \\
-u & --position=ARG & \multirow{4}{2.5in}{Specify antenna position in ECEF (x,y,z) coordinates as "X Y Z". Used to give user-centered data (SV range, azimuth \& elevation).} \\
& & \\
& & \\
& & \\
-s & --start=ARG & \multirow{3}{2.5in}{Specify time to begin analysis as "MO/DD/YYYY HH:MM:SS". The default is the end of the file.} \\
& & \\
& & \\
-e & --end=ARG & \multirow{3}{2.5in}{Specify time to end analysis as "MO/DD/YYYY HH:MM:SS". The default is the beggning of the file.} \\
& & \\
& & \\
-o & --output-filename=ARG & Outputs results to a MATLAB readable file. \\
-t & --time=NUM & \multirow{3}{2.5in}{Specify time increment for ephemeris calculation in seconds. Default is 900 (15 min.)} \\
& & \\
& & \\
\end{tabular}
\end{\outputsize}

\subsection{Examples}
\begin{\outputsize}
\begin{lstlisting}
> WhereSat -b aira1720.06n -p 2 -u "918129.01 -4346070.45 803.18"
  -s "06/21/2006 17:00:00" -e "06/21/2006 20:00:00" -t 1800

 Antenna Position:  918129  -4.34607e+06  803.18
 Navigation File:   aira1720.06n
 Start Time:        06/21/2006 17:00:00
 End Time:          06/21/2006 20:00:00
 PRN:               2

 Prn 2 Earth-fixed position and clock information:

 Date       Time(UTC)   X (meters)          Y (meters)          Z (meters)      
 ===============================================================================
 06/21/2006 18:00:00  12758891.971859      18901201.616227      -14049016.596144
 06/21/2006 18:30:00  12847888.097031      21541501.416411      -9315422.851798 
 06/21/2006 19:00:00  12843576.989405      23087218.618683      -3957280.515764 
 06/21/2006 19:30:00  12450313.769289      23516935.034029      1667186.089065  

  . . .

    Clock Correc (s)
==================
     0.000007
     0.000007
     0.000007
     0.000007

 

 Data for user reference frame:

 Date       Time(UTC)   Azimuth        Elevation      Range to SV (m)
 =====================================================================
 06/21/2006 18:00:00  130.596202      -43.242769      29627531.177821
 06/21/2006 18:30:00  118.680085      -49.681012      29983796.522429
 06/21/2006 19:00:00  102.845663      -53.888528      30169796.433699
 06/21/2006 19:30:00  84.400419       -55.459042      30197072.648367

 Calculated 4 increments for prn 2 .


\end{lstlisting}
\end{\outputsize}

\subsection{Notes}

%\end{document}

