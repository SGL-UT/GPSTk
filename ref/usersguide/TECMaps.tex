%\documentclass{article}
%\usepackage{fancyvrb}
%\usepackage{src/perltex}
%\usepackage{src/xcolor}
%\usepackage{listings}
%\usepackage{longtable}
%\usepackage{multirow}
%\RecustomVerbatimEnvironment{Verbatim}{Verbatim}{frame=single}
\definecolor{console}{rgb}{0.95,0.95,0.95}
\lstset{basicstyle=\ttfamily, columns=flexible, backgroundcolor=\color{console}}

\newcommand{\outputsize}{footnotesize}
\newcommand{\application}[1]{\emph{#1}}
\newcommand{\setconsole}{\lstset{basicstyle=\ttfamily, columns=flexible, backgroundcolor=\color{console}}}
\newcommand{\setfileio}{\lstset{basicstyle=\ttfamily, columns=flexible, backgroundcolor=\color{console}}}

\perlnewcommand{\getuse}[1]
{
        my $command = $_[0];
        $command = $command." > temp 2>&1 |head -n 15";
        system("$command");

	my $counter = 0;
	my $done = 0;
	my $output = "";
        open(input,"temp");

        while(my $line = <input>)
	{
		if($done == 0)
		{
			$output = $output.$line if $counter < 15;
			$output = $output." . . .\n" if $counter >= 15;
			$done = 1 if $counter >= 15;
		}

		$counter = $counter + 1;
	}

        close(input);

        return  "\\begin{\\outputsize}\n" . "\\begin{lstlisting}\n" .
                "> ".$_[0]."\n\n" . $output .
                "\\end{lstlisting}\n" . "\\end{\\outputsize}\n";
}

\perlnewcommand{\getrevision}[1]
{
	my $revision = $_[0];
	$revision =~ m/\$LastChangedRevision: ([^\$]*)/;
	return $1;
}

\perlnewcommand{\entry}[4]
{
	my $output = "$_[0] \& $_[1] \& \\multirow{$_[3]}{2.5in}{$_[2]} \\\\";

	my $cnt = 0;
	while($cnt < ($_[3]-1))
	{
		$output = $output." \& \& \\\\ ";
		$cnt = $cnt + 1;
	}
	
	return $output;
}


%\begin{document}

\index{TECMaps!application writeup}

\section{\emph{TECMaps}}
\subsection{Overview}
 The application will open and read several preprocessed RINEX obs files
 (containing obs types EL,AZ,VR|SR) and use the data to create maps of
 the Total Electron Content (TEC).

\subsection{Usage}
\begin{\outputsize}
\begin{longtable}{lll}
\multicolumn{3}{c}{\application{TECMaps}} \\
\multicolumn{3}{l}{\textbf{Required Arguments}} \\
\entry{}{--input}{Input Rinex obs file name(s)}{1}

& & \\
\multicolumn{3}{l}{\textbf{Optional Arguments}} \\
\entry{}{-f}{file containing more options}{1}
& & \\
\multicolumn{3}{l}{\textbf{Reference station position (one required)}} \\
\entry{}{--RxLLH $<$l,l,h$>$}{Reference site position in geodetic lat, lon (E), ht (deg,deg,m)}{2}
\entry{}{--RxXYZ $<$x,y,z$>$}{Reference site position in ECEF coordinates (m)}{1}
\entry{}{--inputdir}{Path for input file(s)}{1}
& & \\
\multicolumn{3}{l}{\textbf{Ephemeris input}} \\
\entry{}{--navdir}{Path of navigation file(s)}{1}
\entry{}{--nav}{Navigation (Rinex Nav OR SP3) file(s)}{1}
& & \\
\multicolumn{3}{l}{\textbf{Output}} \\
\entry{}{--log}{Output log file name}{1}
& & \\
\multicolumn{3}{l}{\textbf{Time limits}} \\
\entry{}{--BeginTime}{Start time, arg is of the form YYYY,MM,DD,HH,Min,Sec}{2}
\entry{}{--BeginGPSTime}{Start time, arg is of the form GPSweek,GPSsow}{1}
\entry{}{--EndTime}{End time, arg is of the form YYYY,MM,DD,HH,Min,Sec}{2}
\entry{}{--EndGPSTime}{End time, arg is of the form GPSweek,GPSsow}{1}
& & \\
\multicolumn{3}{l}{\textbf{Processing}} \\
\entry{}{--noVTECmap}{Do NOT create the VTEC map.}{1}
\entry{}{--MUFmap}{Create MUF map as well as VTEC map.}{1}
\entry{}{--F0F2map}{Create F0F2 map as well as VTEC map}{1}
\entry{}{--Title1}{Title information}{1}
\entry{}{--Title2}{Second title information}{1}
\entry{}{--BaseName}{Base name for output files (a)}{1}
\entry{}{--DecorrError}{Decorrelation error rate in TECU/1000km (3)}{1}
\entry{}{--Biases}{File containing estimated sat+rx biases (Prgm IonoBias)}{2}
\entry{}{--ElevThresh}{Minimum elevation (6 deg)}{1}
\entry{}{--MinAcqTime}{Minimum acquisition time (0 sec)}{1}
\entry{}{--FlatFit}{Flat fit type (default)}{1}
\entry{}{--LinearFit}{Linear fit type}{1}
\entry{}{--IonoHeight}{Ionosphere height (km)}{1}
& & \\
\multicolumn{3}{l}{\textbf{Grid}} \\
\entry{}{--UniformSpacing}{Grid uniform in space (XYZ) (default)}{1}
\entry{}{--UniformGrid}{Grid uniform in Lat and Lon}{1}
\entry{}{--OutputGrid}{Output the grid to file $<$basename.LL$>$}{1}
\entry{}{--GnuplotOutput}{Write the grid file for gnuplot (default: for Matlab)}{2}
\entry{}{--NumLat}{Number of latitude grid points (40)}{1}
\entry{}{--NumLon}{Number of longitude grid points (40)}{1}
\entry{}{--BeginLat}{Beginning latitude (21 deg)}{1}
\entry{}{--BeginLon}{Beginning longitude (230 deg E)}{1}
\entry{}{--DeltaLat}{Grid spacing in latitude (0.25 deg)}{1}
\entry{}{--DeltaLon}{Grid spacing in longitude (1.0 deg)}{1}
& & \\
\multicolumn{3}{l}{\textbf{Other options}} \\
\entry{}{--XSat}{Exclude this satellite (<sat> may be <system> only)}{2}
& & \\
\multicolumn{3}{l}{\textbf{Help}} \\
\entry{-v}{--verbose}{print extended output info.}{1}
\entry{-d}{--debug}{print extended output info.}{1}
\entry{-h}{--help}{print syntax and summary of input, then quit.}{1}

\end{longtable}
\end{\outputsize}

\subsection{Examples}

\subsection{Notes}
Input is on the command line, or of the same format in a file (-f$<$file$>$).

%\end{document}

