%\documentclass{article}
%\usepackage{fancyvrb}
%\usepackage{src/perltex}
%\usepackage{src/xcolor}
%\usepackage{listings}
%\usepackage{longtable}
%\usepackage{multirow}
%\RecustomVerbatimEnvironment{Verbatim}{Verbatim}{frame=single}
\definecolor{console}{rgb}{0.95,0.95,0.95}
\lstset{basicstyle=\ttfamily, columns=flexible, backgroundcolor=\color{console}}

\newcommand{\outputsize}{footnotesize}
\newcommand{\application}[1]{\emph{#1}}
\newcommand{\setconsole}{\lstset{basicstyle=\ttfamily, columns=flexible, backgroundcolor=\color{console}}}
\newcommand{\setfileio}{\lstset{basicstyle=\ttfamily, columns=flexible, backgroundcolor=\color{console}}}

\perlnewcommand{\getuse}[1]
{
        my $command = $_[0];
        $command = $command." > temp 2>&1 |head -n 15";
        system("$command");

	my $counter = 0;
	my $done = 0;
	my $output = "";
        open(input,"temp");

        while(my $line = <input>)
	{
		if($done == 0)
		{
			$output = $output.$line if $counter < 15;
			$output = $output." . . .\n" if $counter >= 15;
			$done = 1 if $counter >= 15;
		}

		$counter = $counter + 1;
	}

        close(input);

        return  "\\begin{\\outputsize}\n" . "\\begin{lstlisting}\n" .
                "> ".$_[0]."\n\n" . $output .
                "\\end{lstlisting}\n" . "\\end{\\outputsize}\n";
}

\perlnewcommand{\getrevision}[1]
{
	my $revision = $_[0];
	$revision =~ m/\$LastChangedRevision: ([^\$]*)/;
	return $1;
}

\perlnewcommand{\entry}[4]
{
	my $output = "$_[0] \& $_[1] \& \\multirow{$_[3]}{2.5in}{$_[2]} \\\\";

	my $cnt = 0;
	while($cnt < ($_[3]-1))
	{
		$output = $output." \& \& \\\\ ";
		$cnt = $cnt + 1;
	}
	
	return $output;
}


%\begin{document}

\index{ficdiff!application writeup}
\section{\emph{ficdiff}}
\subsection{Overview}
The application compares the contents of two FIC files containing ephemeris data.

\subsection{Usage}
\begin{\outputsize}

\begin{longtable}{lll}
\multicolumn{3}{l}{\textbf{Optional Arguments}} \\
\entry{Short Arg.}{Long Arg.}{Description}{1}
\entry{-d}{--debug}{Increase debug level}{1}
\entry{-v}{--verbose}{Increase verbosity}{1}
\entry{-h}{--help}{Print help usage}{1}
\entry{-t}{--time=TIME}{Start of time range to compare (default = "beginning of time")}{2}
\entry{-e}{--end-time=TIME}{End of time range to compare (default = "end of time")}{2}
& & \\
\multicolumn{3}{c}{ephdiff usage: ficdiff [options] fic1 fic2} \\
\end{longtable}
\end{\outputsize}

\subsection{Examples}
\begin{\outputsize}
\begin{lstlisting}
> ficdiff -t "08/01/2006 12:00:00" fic1 fic2
<FIC BlockNumber: 9
 floats:   139 362 172806 1 1 1386 1 0 0 55296 0 -4.19095e-09 180000 0 . . .
 integers:
 chars:

<FIC BlockNumber: 9
 floats:   139 362 172806 1 1 1386 1 0 0 59392 0 -6.98492e-09 179984 0 . . .
 integers:
 chars:
 . . .

\end{lstlisting}
\end{\outputsize}

\subsection{Notes}

%\end{document}
