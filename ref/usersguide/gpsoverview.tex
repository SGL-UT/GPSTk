\chapter{The Global Positioning System}

The Global Positioning System is actually a U.S. government satellite navigation system that provides a civilian signal. As of this writing, the signal is broadcast simultaneously by a constellation of 29 satellites each with a 12 hour orbit. From any given position on the Earth, 8 to 12 satellites are usually visible at a time.

\section{GPS in a Nutshell}

Each satellite broadcasts spread spectrum signals at 1575.42 and 1227.6 MHz, also known as L1 and L2, respectively. Currently the civil signal is broadcast only on L1. The signal contains two components: a time code and a navigation message. By differencing the received time code with an internal time code, the receiver can determine the distance, or range, that the signal has traveled. This range observation is offset by errors in the (imperfect) receiver clock; therefore it is called a pseudorange. The navigation message contains the satellite ephemeris, which is a numerical model of the satellite's orbit.

GPS receivers record, besides the pseudorange, a measurement called the carrier phase (or just phase); it is also a range observation like the pseudorange, except (1) it has an unknown constant added to it (the phase ambiguity) and (2) it is much smoother (about 100 times less measurement noise than the pseudorange!), which makes it useful for precise positioning. Because of the way it is measured, the phase is subject to random, sudden jumps; these discrete changes always come in multiples of the wavelength of the GPS signal, and are called cycle slips.


\subsection{The Position Solution}

The standard solution for the user location requires a pseudorange measurement and an ephemeris for each satellite in view. At least four measurements are required as there are four unknowns: 3 coordinates of position plus the receiver clock offset. The basic algorithm for the solution is described in the official GPS Interface Control Document, or ICD-GPS-200. The position solution is corrupted due to two sources of error: errors in the observations and errors in the ephemeris.


\subsubsection{Reducing Measurement Errors}

The GPS signal travels through every layer of the Earth's atmosphere. Each layers affects the signal differently. The ionosphere, which is the high-altitude, electrically charged part of the atmosphere, introduces a delay, and therefore a range error, into the signal. The delay is frequency dependent, so it can be directly computed if you have data on both the GPS frequencies. There is also a delay due to the troposphere, the lower part of the atmosphere. This delay too can be modeled and removed. There are many other errors associated with the GPS signal: multipath reflections and relativistic effects are two examples.

More precise applications reduce the effect of error sources by a technique referred to as differential GPS (DGPS). By differencing measurements simultaneously collected by the user and a nearby reference receiver, the errors that are common to both receivers (most of them) are removed. The result of DGPS positioning is a position relative to the reference receiver; adding the reference position to the DGPS solution results in the absolute user position.

The alternative to DGPS is to explicitly model and remove errors. Creating new and robust models of phenomena that effects the GPS signal is an area of active research at ARL:UT and other laboratories. The positioning algorithm can be used to explore such models. Essentially, the basic approach is to turn the positioning algorithm inside out to look at the corrections themselves. For example, observations from a network of receivers can create a global map or model of the ionosphere.


\subsubsection{Improved Ephemeredes}

The GPS position solution can be directly improved by using an improved satellite ephemeris. The U.S National Geospatial-Intelligence Agency (NGA) generates and makes publicly available a number of precise ephemeredes, which are more accurate satellite orbits. Satellite orbits described by the broadcast navigation message have an error on the order of meters; the precise ephemeris has decimeter accuracy. The International GPS Service (IGS) is a global civil cooperative effort that also provides free precise ephemeris products. Global networks of tracking stations produce the observations that make generation of the precise ephemeredes possible.


\section{GPS Data Sources}

GPS observation data from many tracking stations are freely available on the Internet. Many such stations contribute their data to the IGS. In addition, many networks of stations also post their data to the Internet; for example the Australian Regional GPS Network (ARGN) and global cooperatives such as NASA's Crust Dynamics Data Information System (CDIS).

\subsection{GPS File Formats}
Typically GPS observations are recorded in a standardized format developed by and for researchers. Fundamental to this format is the idea that the data should be independent of the type of receiver that collected it. For this reason the format is called Receiver INdependent Exchange, or RINEX. Another format associated with GPS is SP-3, which records the precise ephemeris. The GPSTk supports both RINEX and SP-3 formats.


\subsection{Receiver Protocols}

GPS receivers have become less expensive and more capable over the years, in particular handheld and mobile GPS receivers. The receivers have many features in common. All of the receivers output a position solution every few seconds. All receivers store a list of positions, called waypoints. Many can display maps that can be uploaded. Many can communicate with a PC or handheld to store information or provide position estimates to plotting software.

Typically communication with a PC and other system follows a standard provided by the National Marine Electronics Association called NMEA-0183. NMEA-0183 defines an ASCII based format for communication of position solutions, waypoints and a variety of receiver diagnostics. Here is an example of a line of NMEA data, or sentence:

\begin{verbatim}
$GPGLL,5133.81,N,00042.25,W*75
\end{verbatim}

The data here is a latitude, longitude fix at 51 deg 33.81 min North, 0 deg 42.25 min West; the last part is a checksum.

As a public standard, the NMEA-0183 format has given the user of GPS freedom of choice. NMEA-0183 is the format most typically used by open source applications that utilize receiver-generated positions.

Closed standards are also common. SiRF is a proprietary protocol that is licensed to receiver manufacturers. Many receiver manufacturers implement their own binary protocols. While some of these protocols have been opened to the public, some have been reverse engineered. 




