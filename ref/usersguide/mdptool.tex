%\documentclass{article}
%\usepackage{fancyvrb}
%\usepackage{perltex}
%\usepackage{xcolor}
%\usepackage{listings}
%\usepackage{longtable}
%\usepackage{multirow}
%\RecustomVerbatimEnvironment{Verbatim}{Verbatim}{frame=single}
\definecolor{console}{rgb}{0.95,0.95,0.95}
\lstset{basicstyle=\ttfamily, columns=flexible, backgroundcolor=\color{console}}

\newcommand{\outputsize}{footnotesize}
\newcommand{\application}[1]{\emph{#1}}
\newcommand{\setconsole}{\lstset{basicstyle=\ttfamily, columns=flexible, backgroundcolor=\color{console}}}
\newcommand{\setfileio}{\lstset{basicstyle=\ttfamily, columns=flexible, backgroundcolor=\color{console}}}

\perlnewcommand{\getuse}[1]
{
        my $command = $_[0];
        $command = $command." > temp 2>&1 |head -n 15";
        system("$command");

	my $counter = 0;
	my $done = 0;
	my $output = "";
        open(input,"temp");

        while(my $line = <input>)
	{
		if($done == 0)
		{
			$output = $output.$line if $counter < 15;
			$output = $output." . . .\n" if $counter >= 15;
			$done = 1 if $counter >= 15;
		}

		$counter = $counter + 1;
	}

        close(input);

        return  "\\begin{\\outputsize}\n" . "\\begin{lstlisting}\n" .
                "> ".$_[0]."\n\n" . $output .
                "\\end{lstlisting}\n" . "\\end{\\outputsize}\n";
}

\perlnewcommand{\getrevision}[1]
{
	my $revision = $_[0];
	$revision =~ m/\$LastChangedRevision: ([^\$]*)/;
	return $1;
}

\perlnewcommand{\entry}[4]
{
	my $output = "$_[0] \& $_[1] \& \\multirow{$_[3]}{2.5in}{$_[2]} \\\\";

	my $cnt = 0;
	while($cnt < ($_[3]-1))
	{
		$output = $output." \& \& \\\\ ";
		$cnt = $cnt + 1;
	}
	
	return $output;
}


%\begin{document}

\index{mdptool!application writeup}

\section{\emph{mdptool}}
\subsection{Overview}
The application performs various functions on a stream of MDP data.

\subsection{Usage}
\subsubsection{\emph{mdptool}}
\begin{\outputsize}
\begin{longtable}{lll}
%\multicolumn{3}{c}{\application{mdptool}} \\
\multicolumn{3}{l}{\textbf{Optional Arguments}} \\
\entry{Short Arg.}{Long Arg.}{Description}{1}
\entry{-d}{--debug}{Increase debug level.}{1}
\entry{-v}{--verbose}{Increase verbosity.}{1}
\entry{-h}{--help}{Print help usage.}{1}
\entry{-i}{--input=ARG}{Where to get the MDP data from. The default is to use
                         stdin. If the file name begins with "tcp:" the
                         remainder is assumed to be a hostname[:port] and the
                         source is taken from a tcp socket at this address. If
                         the port number is not specified a default of 8910 is
                         used.}{6}
\entry{}{--output=ARG}{Where to send the output. The default is stdout.}{2}
\entry{-p}{--pvt}{Enable pvt output.}{1}
\entry{-o}{--obs}{Enable obs output.}{1}
\entry{-n}{--nav}{Enable nav output.}{1}
\entry{-t}{--test}{Enable selftest output.}{1}
\entry{-x}{--hex}{Dump all messages in hex.}{1}
\entry{-b}{--bad}{Try to process bad messages also.}{1}
\entry{-a}{--almanac}{Build and process almanacs. Only applies to the nav style.}{2}
\entry{-e}{--ephemeris}{Build and process engineering ephemerides. Only applies to the nav style.}{2}
\entry{}{--min-alm}{This allows a complete almanac to be constructed from fewer than 50 pages.  It is required for Ashtech Z(Y)-12.  The default is to require all 50 pages.}{4}
\entry{-f}{--follow}{Follow the input file as it grows.}{1}
\entry{-s}{--output-style=ARG}{What type of output to produce from the MDP stream.
                         Valid styles are: brief, verbose, table, track, null,
                         mdp, nav, and summary. The default is summary. Some
                         modes aren't quite complete.}{5}
\entry{-l}{--timeSpan=NUM}{How much data to process, in seconds.}{1}
\entry{-m}{--bug-mask=NUM}{What RX bugs: 1 SV count, 2 nav parity/fmt, 4 HOW/hdr time equal.}{2}
\entry{}{--startTime=TIME}{Ignore data before this time. (\%4Y/\%03j/\%02H:\%02M:\%05.2f).}{2}
\entry{}{--stopTime=TIME}{Ignore any data after this time.}{1}
\entry{}{--time-format=ARG}{CommonTime format specifier used for times in the output.  The default is \%4Y \%3j \%02H:\%02M:\%04.1f.}{3}
\end{longtable}
\end{\outputsize}
\subsection{Examples}
\begin{verbatim}
> mdptool -i mdp/85408-2012131-2u.mdp -a
Done processing data.


Header summary:
  Processed 2685 headers.
  First freshness count was d96a
  Last freshness count was  e3e6
  Encountered 0 breaks in the freshness count

Observation Epoch message summary:
No Observation Epoch messages processed.

PVT Solution message summary:
  Pvt data spans 2012/131/00:02:06.0 to 2012/131/17:57:51.0 (17:55:45.0)
  PVT output rate is 1.5 sec.


Navigation Subframe message summary:
  No Navigation Subframe messages processed.

\end{verbatim}


\subsection{Notes}
In the summary mode, the default is to only summarize the observation data above 10 degrees. Increasing the verbosity level will also summarize the data below 10 degrees.

%\end{document}
