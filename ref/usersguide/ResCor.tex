%\documentclass{article}
%\usepackage{fancyvrb}
%\usepackage{src/perltex}
%\usepackage{src/xcolor}
%\usepackage{listings}
%\usepackage{multirow}
%\usepackage{longtable}
%\RecustomVerbatimEnvironment{Verbatim}{Verbatim}{frame=single}
\definecolor{console}{rgb}{0.95,0.95,0.95}
\lstset{basicstyle=\ttfamily, columns=flexible, backgroundcolor=\color{console}}

\newcommand{\outputsize}{footnotesize}
\newcommand{\application}[1]{\emph{#1}}
\newcommand{\setconsole}{\lstset{basicstyle=\ttfamily, columns=flexible, backgroundcolor=\color{console}}}
\newcommand{\setfileio}{\lstset{basicstyle=\ttfamily, columns=flexible, backgroundcolor=\color{console}}}

\perlnewcommand{\getuse}[1]
{
        my $command = $_[0];
        $command = $command." > temp 2>&1 |head -n 15";
        system("$command");

	my $counter = 0;
	my $done = 0;
	my $output = "";
        open(input,"temp");

        while(my $line = <input>)
	{
		if($done == 0)
		{
			$output = $output.$line if $counter < 15;
			$output = $output." . . .\n" if $counter >= 15;
			$done = 1 if $counter >= 15;
		}

		$counter = $counter + 1;
	}

        close(input);

        return  "\\begin{\\outputsize}\n" . "\\begin{lstlisting}\n" .
                "> ".$_[0]."\n\n" . $output .
                "\\end{lstlisting}\n" . "\\end{\\outputsize}\n";
}

\perlnewcommand{\getrevision}[1]
{
	my $revision = $_[0];
	$revision =~ m/\$LastChangedRevision: ([^\$]*)/;
	return $1;
}

\perlnewcommand{\entry}[4]
{
	my $output = "$_[0] \& $_[1] \& \\multirow{$_[3]}{2.5in}{$_[2]} \\\\";

	my $cnt = 0;
	while($cnt < ($_[3]-1))
	{
		$output = $output." \& \& \\\\ ";
		$cnt = $cnt + 1;
	}
	
	return $output;
}


%\begin{document}

\index{ResCor!application writeup}
\section{\emph{ResCor}}
\subsection{Overview}
The application will open and read a single RINEX observation file, apply editing commands
   using the RinexEditor package, compute any of several residuals and corrections and
   register extended RINEX observation types for them, and then write the edited data,
   along with the new extended observation types, to an output RINEX observation file.

\subsection{Usage}
%Required arguments:
% -IF and -OF (RinexEditor commands, see below) are required argument
\begin{\outputsize}

\begin{longtable}{lll}
\multicolumn{3}{c}{\application{ResCor}} \\
\multicolumn{3}{l}{\textbf{Configuration Arguments}} \\
\entry{Short Arg.}{Long Arg.}{Description}{1}
\entry{-f$<$file$>$}{}{File containing more options}{1}
\entry{}{--nav $<$file$>$}{Navigation (Rinex Nav OR SP3) file(s)}{1}
\entry{}{--navdir $<$dir$>$}{Directory of navigation file(s)}{1}
& & \\
\multicolumn{3}{l}{\textbf{Reference position input}} \\
\entry{}{--RxLLH $<$l,l,h$>$}{1.Receiver position (static) in geodetic lat, lon(E), ht (deg,deg,m)}{2}
\entry{}{--RxXYZ $<$x,y,z$>$}{2.Receiver position (static) in ECEF coordinates (m)}{2}
\entry{}{--Rxhere}{3.Reference site positions(time) from this file (i.e. -IF$<$RinexFile$>$)}{2}
\entry{}{--RxRinex $<$fn$>$}{4.Reference site positions(time) from another Rinex file named $<$fn$>$}{2}
\entry{}{--RxFlat $<$fn$>$}{5.Reference site positions and times given in a flat file named $<$fn$>$}{2}
\entry{}{--Rxhelp}{(Enter --Rxhelp for a description of the -RxFlat file format)}{2}
\entry{}{--RAIM}{6.Reference site positions computed via RAIM (requires P1,P2,EP)}{2}
\entry{}{}{NB the following two options apply only if --RAIM is found)}{2}
\entry{}{--noRAIMedit}{Do not edit data based on RAIM solution}{1}
\entry{}{--RAIMhead}{Output average RAIM solution to Rinex header (if -HDf also appears)}{2}
\entry{}{--noRefout}{Do not output reference solution to Rinex}{1}
& & \\
\multicolumn{3}{l}{\textbf{Residual/Correction computation}} \\
\entry{}{--debias $<$OT,l$>$}{Debias new output type $<$OT$>$; trigger a bias reset with limit $<$l$>$}{2}
\entry{}{--Callow}{Allow C1 to replace P1 when P1 is not available}{1}
\entry{}{--Cforce}{Force C/A code pseudorange C1 to replace P1}{1}
\entry{}{--IonoHt $<$ht$>$}{Height of ionosphere in km (default 400) (needed for LA,LO,VR,VP)}{2}
\entry{}{--SVonly $<$prn$>$}{Process this satellite ONLY}{1}
& & \\
\multicolumn{3}{l}{\textbf{Output files}} \\
\entry{}{--Log $<$file$>$}{Output log file name (rc.log)}{1}
\entry{}{--Err $<$file$>$}{Output error file name (rc.err)}{1}
& & \\
\multicolumn{3}{l}{\textbf{Help}} \\
\entry{}{--verbose}{Print extended output}{1}
\entry{}{--debug}{Print debugging information.}{1}
\entry{-h}{--help}{Print syntax and quit.}{1}
\end{longtable}

\begin{verbatim}
Rinex Editor commands:
 ==============================================================================
 Commands begin with a '-' or '/', followed by an identifier, then data fields.
 Fields beyond the initial 2- or 3-character identifier are comma delimited.
 <SV> gives a satellite; SV=<PRN><System(optional)> eg. 19G or 19 = PRN 19 GPS.
 <System> is a single character (G=GPS, R=GLONASS, T=Transit, S=Geosynchronous).
 <OT> gives a Rinex observation type, e.g. L1 or P2 (case sensitive).
 <time> gives a time; time=<week,sow> OR time=<year,mon,day,hour,min,second>.
\end{verbatim}

\begin{longtable}{lll}
\multicolumn{3}{l}{\textbf{File I/O}} \\
\entry{-IF$<$file$>$}{}{Input Rinex observation file name (required)}{1}
\entry{-ID$<$dir$>$}{}{Directory in which to find input file}{1}
\entry{-OF$<$file$>$}{}{Output Rinex file name (required, or -OF$<$file$>$,$<$time$>$)}{2}
\entry{-OF$<$f$>$,$<$time$>$}{}{At time=$<$time$>$, close output file and open another named $<$f$>$}{2}
\entry{-OD$<$dir$>$}{}{Directory in which to put output file(s)}{1}
& & \\
\multicolumn{3}{l}{\textbf{Output Rinex header fields}} \\
\entry{-HDf}{}{If present, fill optional records in the output Rinex header (NB EditObs() and EditFile() will do this, but NOT EditHeader().)}{3}
\entry{-HDp$<$program$>$}{}{Set output Rinex header 'program' field}{1}
%\entry{-HDr<run_by>}{}{Set output Rinex header 'run by' field}{2}
\entry{-HDo$<$observer$>$}{}{Set output Rinex header 'observer' field}{1}
\entry{-HDa$<$agency$>$}{}{Set output Rinex header 'agency' field}{1}
\entry{-HDm$<$marker$>$}{}{Set output Rinex header 'marker' field}{1}
\entry{-HDn$<$number$>$}{}{Set output Rinex header 'number' field}{1}
\entry{-HDc$<$comment$>$}{}{Add comment to output Rinex header (more than one allowed).}{2}
\entry{-HDdc}{}{Delete all comments in output Rinex header}{1}
\entry{}{}{(NB -HDdc cannot delete comments created by *subsequent* -HDc commands)}{2}
\entry{-AO$<$OT$>$}{}{Add observation type OT to header and observation data}{2}
& & \\
\multicolumn{3}{l}{\textbf{General edit commands}} \\
\entry{-TB$<$time$>$}{}{Begin time: reject data before this time (also used for decimation)}{2}
\entry{-TE$<$time$>$}{}{End   time: reject data after this time}{1}
\entry{-TT$<$dt$>$}{}{Tolerance in comparing times, in seconds (default=1ms)}{2}
\entry{-TN$<$dt$>$}{}{Decimate data to epochs = Begin + integer*dt (within tolerance)}{2}
\end{longtable}
\begin{verbatim}
 Specific edit commands:
 =======================
 (Generally each '+' command (e.g DA+,<time>) has a corresponding '-' command,
  and vice-versa; if not, End-of-file or Begin-of-file is assumed.
  Note commands at one time are applied AFTER other commands of the same type.)
\end{verbatim}

\begin{longtable}{lll}
\multicolumn{3}{l}{\textbf{Delete Command}} \\
\entry{-DA+$<$time$>$}{}{Delete all data beginning at this time}{1}
\entry{-DA-$<$time$>$}{}{Stop deleting data at this time}{1}
\entry{-DO$<$OT$>$}{}{Delete observation type OT entirely (including in header)}{3}
\entry{-DS$<$SV$>$}{}{Delete all data for satellite SV entirely (SV may be system only)}{2}
\entry{-DS$<$SV$>$,$<$time$>$}{}{Delete all data for satellite SV at this single time (only)}{2}
\entry{-DA-$<$time$>$}{}{Stop deleting data at this time}{1}
\entry{-DO$<$OT$>$}{}{Delete observation type OT entirely (including in header)}{1}
\entry{-DS$<$SV$>$}{}{Delete all data for satellite SV entirely (SV may be system only)}{2}
\entry{-DS$<$SV$>$,$<$time$>$}{}{Delete all data for satellite SV at this single time (only)}{2}
\entry{-DS+$<$SV$>$,$<$time$>$}{}{Delete all data for satellite SV beginning at this time}{2}
\entry{-DS-$<$SV$>$,$<$time$>$}{}{Stop deleting all data for satellite SV at this time}{2}
\entry{}{}{(NB DS commands with SV=system (only) delete all satellites of that system.)}{2}
\entry{-DD$<$SV,OT,t$>$}{}{Delete a single Rinex data(SV,OT,t) at time $<$t$>$}{2}
\entry{-DD+$<$SV,OT,t$>$}{}{Delete all (SV,OT) data, beginning at time $<$t$>$}{1}
\entry{-DD-$<$SV,OT,t$>$}{}{Stop deleting all (SV,OT) data at time $<$t$>$}{1}
\entry{}{}{(NB deleting data for one OT means setting it to zero - here and in Rinex)}{2}
& & \\
\multicolumn{3}{l}{\textbf{Set Commands}} \\
\entry{-SD$<$SV,OT,t,d$>$}{}{Set data(SV,OT,t) to $<$d$>$ at time $<$t$>$}{1}
\entry{-SS$<$SV,OT,t,s$>$}{}{Set ssi(SV,OT,t) to $<$s$>$ at time $<$t$>$}{1}
\entry{-SL+$<$SV,OT,t,l$>$}{}{Set all lli(SV,OT,t) to $<$l$>$ at time $<$t$>$}{1}
\entry{-SL-$<$SV,OT,t,l$>$}{}{Stop setting lli(SV,OT,t) to $<$l$>$ at time $<$t$>$ (',$<$l$>$' is optional)}{2}
\entry{-SL$<$SV,OT,t,l$>$}{}{Set lli(SV,OT,t) to $<$l$>$ at the single time $<$t$>$ (only)}{2}
\entry{}{}{(NB SL commands with SV=system (only) modify all satellites of that system.)}{2}
& & \\
\multicolumn{3}{l}{\textbf{Bias Commands}} \\
\entry{}{}{(NB. BD commands apply only when data is non-zero, unless -BZ appears)}{2}
\entry{-BZ}{}{Apply bias data commands (BD) even when data is zero}{2}
\entry{-BD$<$SV,OT,t,d$>$}{}{Add the value of $<$d$>$ to data(SV,OT,t) at time $<$t$>$}{2}
\entry{-BD+$<$SV,OT,t,d$>$}{}{Add value of $<$d$>$ to data(SV,OT) beginning at time $<$t$>$}{2}
\entry{-BD-$<$SV,OT,t,d$>$}{}{Stop adding $<$d$>$ to data(SV,OT) at time $<$t$>$ (',$<$d$>$' optional)}{2}
\entry{-BS$<$SV,OT,t,s$>$}{}{Add the value of $<$s$>$ to ssi(SV,OT,t) at time $<$t$>$}{2}
\entry{-BL$<$SV,OT,t,l$>$}{}{Add the value of $<$l$>$ to lli(SV,OT,t) at time $<$t$>$}{2}
\end{longtable}

\begin{verbatim}
The list of available extended Rinex obs types:
  OT Description          Units     Required input (EP=ephemeris,PS=Rx Position)
  ER Ephemeris range      meters                 EP PS
  RI Iono Delay, Range    meters              P1
  PI Iono Delay, Phase    meters     L1 L2
  TR Tropospheric Delay   meters                 EP PS
  RL Relativity Correct.  meters                 EP
  SC SV Clock Bias        meters                 EP
  EL Elevation Angle      degrees                EP PS
  AZ Azimuth Angle        degrees                EP PS
  SR Slant TEC (PR)       TECU                P1
  SP Slant TEC (Ph)       TECU       L1 L2
  VR Vertical TEC (PR)    TECU                P1 EP PS
  VP Vertical TEC (Ph)    TECU       L1 L2       EP PS
  LA Lat Iono Intercept   degrees                EP PS
  LO Lon Iono Intercept   degrees                EP PS
  P3 TFC(IF) Pseudorange  meters              P1
  L3 TFC(IF) Phase        meters     L1 L2
  P4 GeoFree Pseudorange  meters              P1
  L4 GeoFree Phase        meters     L1 L2
  P5 WideLane Pseudorange meters              P1
  L5 WideLane Phase       meters     L1 L2
  MP Multipath (=M3)      meters     L1 L2    P1
  M1 L1 Range minus Phase meters     L1       P1
  M2 L2 Range minus Phase meters        L2
  M3 IF Range minus Phase meters     L1 L2    P1
  M4 GF Range minus Phase meters     L1 L2    P1
  M5 WL Range minus Phase meters     L1 L2    P1
  XR Non-dispersive Range meters     L1 L2    P1
  XI Ionospheric delay    meters     L1 L2    P1
  X1 Range Error L1       meters     L1 L2    P1
  X2 Range Error L2       meters     L1 L2    P1
  SX Satellite ECEF-X     meters                 EP
  SY Satellite ECEF-Y     meters                 EP
  SZ Satellite ECEF-Z     meters                 EP
\end{verbatim}

\end{\outputsize}

\subsection{Examples}

\subsection{Notes}

%\end{document}
