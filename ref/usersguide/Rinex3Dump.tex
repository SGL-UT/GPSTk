%\documentclass{article}
%\usepackage{fancyvrb}
%\usepackage{src/perltex}
%\usepackage{src/xcolor}
%\usepackage{listings}
%\usepackage{longtable}
%\usepackage{multirow}
%\RecustomVerbatimEnvironment{Verbatim}{Verbatim}{frame=single}
\definecolor{console}{rgb}{0.95,0.95,0.95}
\lstset{basicstyle=\ttfamily, columns=flexible, backgroundcolor=\color{console}}

\newcommand{\outputsize}{footnotesize}
\newcommand{\application}[1]{\emph{#1}}
\newcommand{\setconsole}{\lstset{basicstyle=\ttfamily, columns=flexible, backgroundcolor=\color{console}}}
\newcommand{\setfileio}{\lstset{basicstyle=\ttfamily, columns=flexible, backgroundcolor=\color{console}}}

\perlnewcommand{\getuse}[1]
{
        my $command = $_[0];
        $command = $command." > temp 2>&1 |head -n 15";
        system("$command");

	my $counter = 0;
	my $done = 0;
	my $output = "";
        open(input,"temp");

        while(my $line = <input>)
	{
		if($done == 0)
		{
			$output = $output.$line if $counter < 15;
			$output = $output." . . .\n" if $counter >= 15;
			$done = 1 if $counter >= 15;
		}

		$counter = $counter + 1;
	}

        close(input);

        return  "\\begin{\\outputsize}\n" . "\\begin{lstlisting}\n" .
                "> ".$_[0]."\n\n" . $output .
                "\\end{lstlisting}\n" . "\\end{\\outputsize}\n";
}

\perlnewcommand{\getrevision}[1]
{
	my $revision = $_[0];
	$revision =~ m/\$LastChangedRevision: ([^\$]*)/;
	return $1;
}

\perlnewcommand{\entry}[4]
{
	my $output = "$_[0] \& $_[1] \& \\multirow{$_[3]}{2.5in}{$_[2]} \\\\";

	my $cnt = 0;
	while($cnt < ($_[3]-1))
	{
		$output = $output." \& \& \\\\ ";
		$cnt = $cnt + 1;
	}
	
	return $output;
}


%\begin{document}

\index{Rinex3Dump!application writeup}
\section{\emph{Rinex3Dump}}
\subsection{Overview}
The application reads a RINEX3 file and dumps the obervation data for the given satellite(s) to the standard output.

\subsection{Usage}
\begin{\outputsize}
\begin{longtable}{lll}
\multicolumn{3}{c}{\application{Rinex3Dump}} \\
\multicolumn{3}{l}{\textbf{Optional Arguments}} \\
\entry{Short Arg.}{Long Arg.}{Description}{1}
\entry{-f}{--file $<$file$>$}{Input file is a RINEX observation file. This option may be repeated.
	Optional, but may be needed in case of ambiguity.}{3}
\entry{}{--format $<$format$>$}{The format of the time output. Default is \%4F \%10.3g.}{2}
\entry{-h}{--help}{Prints out this help and exits.}{1}
\entry{-n}{--num}{Make output purely numeric, ie. no header, no system char on satellites.}{2}
\entry{-o}{--obs $<$obs$>$}{RINEX observation type (eg. C1C) found in the file header.
	Optional, but may be needed in case of ambiguity.}{3}
\entry{-p}{--pos}{Only output positions from aux headers, ie. sat and obs are ignored.}{2}
\entry{-s}{--sat $<$sat$>$}{RINEX satellite ID (eg. For GPS PRN 31, $<$sat$>$ = G01).
	Optional, but may be needed in case of ambiguity.}{3}
\entry{-v}{--verbose}{Prints out verbose output.}{1}
\end{longtable}
\begin{verbatim}
Rinex3Dump usage: Rinex3Dump [-n] <rinex obs file> [<satellite(s)> <obstype(s)>] 

The optional argument -n tells Rinex3Dump its output should be purely numeric.
\end{verbatim}
\end{\outputsize}


\subsection{Notes}
MATLAB and Octave can read the purely numeric output.

%\end{document}
