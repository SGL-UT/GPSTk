%\documentclass{article}
%\usepackage{fancyvrb}
%\usepackage{perltex}
%\usepackage{xcolor}
%\usepackage{listings}
%\usepackage{longtable}
%\usepackage{multirow}
%\RecustomVerbatimEnvironment{Verbatim}{Verbatim}{frame=single}
\definecolor{console}{rgb}{0.95,0.95,0.95}
\lstset{basicstyle=\ttfamily, columns=flexible, backgroundcolor=\color{console}}

\newcommand{\outputsize}{footnotesize}
\newcommand{\application}[1]{\emph{#1}}
\newcommand{\setconsole}{\lstset{basicstyle=\ttfamily, columns=flexible, backgroundcolor=\color{console}}}
\newcommand{\setfileio}{\lstset{basicstyle=\ttfamily, columns=flexible, backgroundcolor=\color{console}}}

\perlnewcommand{\getuse}[1]
{
        my $command = $_[0];
        $command = $command." > temp 2>&1 |head -n 15";
        system("$command");

	my $counter = 0;
	my $done = 0;
	my $output = "";
        open(input,"temp");

        while(my $line = <input>)
	{
		if($done == 0)
		{
			$output = $output.$line if $counter < 15;
			$output = $output." . . .\n" if $counter >= 15;
			$done = 1 if $counter >= 15;
		}

		$counter = $counter + 1;
	}

        close(input);

        return  "\\begin{\\outputsize}\n" . "\\begin{lstlisting}\n" .
                "> ".$_[0]."\n\n" . $output .
                "\\end{lstlisting}\n" . "\\end{\\outputsize}\n";
}

\perlnewcommand{\getrevision}[1]
{
	my $revision = $_[0];
	$revision =~ m/\$LastChangedRevision: ([^\$]*)/;
	return $1;
}

\perlnewcommand{\entry}[4]
{
	my $output = "$_[0] \& $_[1] \& \\multirow{$_[3]}{2.5in}{$_[2]} \\\\";

	my $cnt = 0;
	while($cnt < ($_[3]-1))
	{
		$output = $output." \& \& \\\\ ";
		$cnt = $cnt + 1;
	}
	
	return $output;
}


%\begin{document}

\index{RinDump!application writeup}
\section{\emph{RinDump}}
\subsection{Overview}
The application reads a RINEX file and dumps the obervation types in columns.  Output is to the screen, with one time tag and one satellite per line.

\subsection{Usage}
\begin{\outputsize}
\begin{longtable}{lll}
\multicolumn{3}{c}{\application{RinexDump}}\\
\multicolumn{3}{l}{\textbf{Optional Arguments}}\\
\entry{Short Arg.}{Long Arg.}{Description}{1}
\entry{}{--pos}{Output only positions from aux headers; sat and obs are ignored.}{2}
\entry{-n}{--num}{Make output purely numeric (no header, no system char on sats).}{2}
\entry{}{--format $<$file$>$}{Output times in CommonTime format (Default: \%4F \%10.3g).}{2}
\entry{}{--file $<$file$>$}{RINEX observation file; this option may be repeated.}{2}
\entry{}{--obs $<$obs$>$}{RINEX observation type, found in file header.}{1}
\entry{}{--sat $<$sat$>$}{RINEX satellite ID (e.g. G31 for GPS PRN 31).}{1}
\entry{-h}{--help}{Print this and quit.}{1}
\end{longtable}

\begin{verbatim}
RinexDump usage: RinexDump [-n] <rinex obs file> [<satellite(s)> <obstype(s)>] 

The optional argument -n tells RinexDump its output should be purely numeric.
\end{verbatim}
\end{\outputsize}

\subsection{Examples}
\begin{\outputsize}
\begin{lstlisting}
> RinexDump algo1580.06o 3 4 5

# Rinexdump file: algo1580.06o Satellites: G03 G04 G05 Observations: ALL
# Week  GPS_sow Sat            L1 L S            L2 L S            C1 L S
1378 259200.000 G03  -3843024.647 0 3  -2994560.443 0 1  23796436.087 0 0
1378 259230.000 G03  -3954052.735 0 3  -3081075.654 0 2  23775308.750 0 0
1378 259260.000 G03  -4064994.465 0 2  -3167523.561 0 3  23754197.617 0 0

 . . .

          P2 L S            P1 L S            S1 L S            S2 L S
23796439.457 0 0  23796436.350 0 0        21.100 0 0        11.000 0 0
23775311.168 0 0  23775308.182 0 0        22.100 0 0        17.800 0 0
23754199.648 0 0  23754196.550 0 0        17.000 0 0        18.600 0 0

 . . .
\end{lstlisting}
\end{\outputsize}

\subsection{Notes}
MATLAB and Octave can read the purely numeric output.

%\end{document}
