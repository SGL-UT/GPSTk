%\documentclass{article}
%\usepackage{fancyvrb}
%\usepackage{src/perltex}
%\usepackage{src/xcolor}
%\usepackage{listings}
%\usepackage{longtable}
%\usepackage{multirow}
%\RecustomVerbatimEnvironment{Verbatim}{Verbatim}{frame=single}
\definecolor{console}{rgb}{0.95,0.95,0.95}
\lstset{basicstyle=\ttfamily, columns=flexible, backgroundcolor=\color{console}}

\newcommand{\outputsize}{footnotesize}
\newcommand{\application}[1]{\emph{#1}}
\newcommand{\setconsole}{\lstset{basicstyle=\ttfamily, columns=flexible, backgroundcolor=\color{console}}}
\newcommand{\setfileio}{\lstset{basicstyle=\ttfamily, columns=flexible, backgroundcolor=\color{console}}}

\perlnewcommand{\getuse}[1]
{
        my $command = $_[0];
        $command = $command." > temp 2>&1 |head -n 15";
        system("$command");

	my $counter = 0;
	my $done = 0;
	my $output = "";
        open(input,"temp");

        while(my $line = <input>)
	{
		if($done == 0)
		{
			$output = $output.$line if $counter < 15;
			$output = $output." . . .\n" if $counter >= 15;
			$done = 1 if $counter >= 15;
		}

		$counter = $counter + 1;
	}

        close(input);

        return  "\\begin{\\outputsize}\n" . "\\begin{lstlisting}\n" .
                "> ".$_[0]."\n\n" . $output .
                "\\end{lstlisting}\n" . "\\end{\\outputsize}\n";
}

\perlnewcommand{\getrevision}[1]
{
	my $revision = $_[0];
	$revision =~ m/\$LastChangedRevision: ([^\$]*)/;
	return $1;
}

\perlnewcommand{\entry}[4]
{
	my $output = "$_[0] \& $_[1] \& \\multirow{$_[3]}{2.5in}{$_[2]} \\\\";

	my $cnt = 0;
	while($cnt < ($_[3]-1))
	{
		$output = $output." \& \& \\\\ ";
		$cnt = $cnt + 1;
	}
	
	return $output;
}


%\begin{document}

\index{rinexpvt!application writeup}
\section{\emph{rinexpvt}}
\subsection{Overview}
The application generates a user position based on RINEX observation data with the option of including navigation and meteriological data to aid error correction.

\subsection{Usage}
\begin{\outputsize}
\begin{longtable}{lll}
\multicolumn{3}{c}{\application{navdmp}} \\
\multicolumn{3}{l}{\textbf{Required Arguments}} \\
\entry{Short Arg.}{Long Arg.}{Description}{1}
\entry{-o}{--obs-file=ARG}{RINEX obs file}{1}
& & \\

\multicolumn{3}{l}{\textbf{Optional Arguments}} \\
\entry{Short Arg.}{Long Arg.}{Description}{1}
\entry{-d}{--debug}{Increase debug level}{1}
\entry{-v}{--verbose}{Increase verbosity}{1}
\entry{-h}{--help}{Print help usage}{1}
\entry{-n}{--nav-file=ARG}{RINEX Nav file. Required for single frequency ionosphere correction.}{2}
\entry{-p}{--pe-file=ARG}{SP3 Precise Ephemeris File. Repeat this for each input file.}{2}
\entry{-m}{--met-file=ARG}{RINEX Met File}{1}
\entry{-t}{--time-format=ARG}{Alternate time format string.}{1}
\entry{-e}{--enu=ARG}{Use the following as origin to solve for East/North/Up coordinates, formatted as a string: "X Y Z"}{3}
\entry{-l}{--elevation-mask=ARG}{Elevation mask (degrees)}{1}
\entry{-s}{--single-frequency}{Use only C1 (SPS)}{1}
\entry{-f}{--dual-frequency}{Use only P1 and P2 (PPS)}{1}
\entry{-i}{--no-ionosphere}{Do NOT correct for ionosphere delay.}{1}
\entry{-x}{--no-closest-ephemeris}{Allow ephemeris use outside of fit interval.}{1}
\entry{-c}{--no-carrier-smoothing}{Do NOT use carrier phase smoothing.}{1}
\end{longtable}
\end{\outputsize}

\subsection{Examples}
\getuse{rinexpvt -o arl2800.06o -n arl2800.06n}

\getuse{rinexpvt -o arl2800.06o -n arl2800.06n -m arl2800.06m}

\subsection{Notes}
Though not stated in the required options lists either a RINEX navigation file or an SP3 Precise Ephemeris File is needed, using the -n or -p option respectively. When using precise ephemeris 3 files must be included, the previous day, the current day and the next day. 

%\end{document}

