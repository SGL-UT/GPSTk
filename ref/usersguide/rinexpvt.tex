\chapter{RINEXPVT}

By R. Benjamin Harris

\section{Overview}

The most common application of GPS is to simply determine the position of the
user. The \texttt{rinexpvt} application is a GPSTk [1] based application that
generates user positions from pseudoranges recorded in the RINEX [2] format.
One user position, or PVT, is generated per epoch of observation. No smoothing
is applied to the pseudoranges, nor are the solutions filtered. A number of
error models are applied to the pseudoranges before the generating the
position calcuation, such as atmospheric delay. The user can select an
elevation mask for satellites. Also only healthy satellites, as defined by the
broadcast ephemeris, are used in the solution. The user can control which pseudoranges
are used, and some of the corrections applied to them.

\section{Synopsis}

The user executes \texttt{rinexpvt} from the command line. The processing
performed by the \texttt{rinexpvt} is specified through command line
arguments. The full set of arguments is defined below. This list can be
duplicated by running \texttt{rinexpvt -h} on the command line.

\small
\begin{singlespace}
\begin{verbatim}
./rinexpvt -h
Usage: rinexpvt [OPTION] ...
GPSTk PVT Generator

This application generates user positions based on RINEX observations.

NOTE: Although the -n and -p arguments appear as optional below, one of the two
must be used. An ephemeris source must be specified.

Required arguments:
  -o, --obs-file=ARG         RINEX Obs File.

Optional arguments:
  -d, --debug                Increase debug level
  -v, --verbose              Increase verbosity
  -h, --help                 Print help usage
  -n, --nav-file=ARG         RINEX Nav File. Required for single frequency ionosphere correction.
  -p, --pe-file=ARG          SP3 Precise Ephemeris File. Repeat this for each input file.
  -m, --met-file=ARG         RINEX Met File.
  -t, --time-format=ARG      Alternate time format string.
  -e, --enu=ARG              Use the following as origin to solve 
                             for East/North/Up coordinates, 
                             formatted as a string: "X Y Z"
  -l, --elevation-mask=ARG   Elevation mask (degrees).
  -s, --single-frequency     Use only C1 (SPS)
  -f, --dual-frequency       Use only P1 and P2 (PPS)
  -i, --no-ionosphere        Do NOT correct for ionosphere delay.
  -x, --no-closest-ephemeris Allow ephemeris use outside of fit interval.
  -c, --no-carrier-smoothing Do NOT use carrier phase smoothing.
\end{verbatim}
\end{singlespace}
\normalsize

\section{Detailed Description}

\subsection{Observation Model}

The user position is related to the pseudorange observation through the
following formula [3, 4].


\begin{equation}
  \rho = \sqrt{( x_u - x_s )^2 + ( y_u - y_s )^2 + ( z_u - z_s^{^{}})^2} + c \delta t + t + i + \nu + m + \epsilon
\end{equation}
where \begin{tabular}{l}
  $\rho$ is the pseudorange measurement\\
  $x, y$, and $z$ represent Cartesian coordinates\\
  $u$ is the user position at time of reception\\
  $s$ is the satellite position at the time of transmission\\
  $c \delta t$ is the clock offset between the user and spacecraft clocks\\
  $t$ is the delay due to the troposphere\\
  $i$is delay due to the ionosphere\\
  $\nu$ is relativistic delay\\
  $m$ is multipath delay\\
  $\epsilon$ is thermal measurement noise
\end{tabular}

For each satellite in view at a given epoch, one independent
relation can be formed. When the satellite position is known as a function of
time, and atmospheric delays have been estimated, then unknown terms are user
position, at $x_s, y_s$, and $z_s,$and the clock offset, $c \delta t$. If more
than four such observations are available for a given epoch, then gradient
search methods are be used in combination with least squares to solve for the
user position [3, 4, 5].

\subsection{Satellite Position Models}

Satellite positions are computed as a function of time by one of two methods.
The first method applies modified Keplerian parameters found in the broadcast
ephemeris, as defined by the ICD-GPS-200 [3, 4, 6]. The second method is by
Lagrange interpolation of precise ephemerides [3].

\subsection{Delay Models}

Many of the delays found in Eq. 1 are modeled within \texttt{rinexpvt}. The
delay due to special relativity can be computed directly from satellite
position and velocity. The troposphere delay can be estimated using
meteorological observations. Finally, the ionosphere delay can be computed
using additional range observation or using a reference model.

The net effect due to special relativistic delay is frequently modeled within
receivers using the following equation [4, 6]
\begin{equation}
  \delta t' = 2 \frac{\vec{r} \bullet \vec{v}}{c^2}
\end{equation}
where $\vec{r}$ is the Earth centered, Earth fixed (ECEF) position vector,
$\vec{v}$ is the ECEF velocity vector and $c$ is the speed of light.

If observations from more than one frequency are available for an epoch, then
the ionosphere delay $i$ is estimated using the following linear relationship
[5].
\begin{equation}
  i \approx \frac{f_2^2}{f_2^2 - f_1^2} ( P_1 - P_2 )
\end{equation}
where $P_1 $is the pseudorange measurement on L1 and $P_2$ is the pseudorange
measurement on L2. If dual frequency measurements are not available, then
ionosphere delay is estimated using the Klobuchar model [2]. Note that the
Klobuchar model parameters are found in the navigation message, so to apply
the model, \texttt{rinexpvt} must be provided a RINEX nav file.

The troposphere delay is estimated using meteorological observations. The
modified Hopfield model is used within \texttt{rinexpvt} to model this form of
error.[7] If no actual weather measurements are provided, then a default
weather condition is assumed: 20 degrees ceslius, 1000 millibars and 50
percent humidity.

\section{Examples and Usage Notes}

This section contains a number of practical examples in the use of
\texttt{rinexpvt}. In each subsection there is a brief description of the
desired processing, as well as a snapshot of a shell session as a
demonstration. All of the example files used are distributed with the
application.

\subsection{Generating Positions in WGS 84 Coordinates}

By default, \texttt{rinexpvt} generates the user position in the Cartesian,
WGS reference frame. The user needs only supply observations and a source of
ephemerides. If a RINEX meteorological file is provided, then troposphere
delays are modeled. Otherwise the troposphere delay is modeled using a
standard temperature of 20 celsius, . The following example demonstrates the
most basic processing provided by \texttt{rinexpvt}.

\small
\begin{singlespace}
\begin{verbatim}
rinexpvt -o usno0200.05o -n brdc0200.05n -m usno0200.05

2005 1 20 00 00 0.000000 1112192.67926 -4842951.98205 3985348.06329
2005 1 20 00 00 30.000000 1112188.65938 -4842953.48346 3985351.48398
2005 1 20 00 01 0.000000 1112189.48576 -4842957.45711 3985356.92698
2005 1 20 00 01 30.000000 1112191.15384 -4842957.53284 3985355.11895
2005 1 20 00 02 0.000000 1112191.1508 -4842955.89459 3985352.76549
2005 1 20 00 02 30.000000 1112190.99828 -4842954.61737 3985352.3681
2005 1 20 00 03 0.000000 1112189.92412 -4842954.29518 3985348.72842
2005 1 20 00 03 30.000000 1112189.16937 -4842954.31307 3985351.92146
2005 1 20 00 04 0.000000 1112191.62124 -4842955.6613 3985354.82972
2005 1 20 00 04 30.000000 1112188.66733 -4842953.49038 3985348.04322
\end{verbatim}
\end{singlespace}
\normalsize

\subsection{Generating Positions in East/North/Up Coordinates}

The user may wish to transform the results of the position calculation to
local, or topocentric, reference frame. The new coordinates, still Cartesian,
refer to the cardinal directions: East, North and Up. The positions calculated
by \texttt{rinexpvt} can be tranformed to a topocentric origin using the
\texttt{-e} option. The argument to this option is a single string, with three
numerical entries for the origin of the topocentric system. Often, within
RINEX observation files, the header entry ``APPROX POS XYZ'' is a recent
surveyed origin of the receiver and forms a useful origin for a topocentric
system. In the following example the topocentric transformation is applied to
the results from the previous subsection.

\small
\begin{singlespace}
\begin{verbatim}
grep \"APPROX \" usno0200.05o
  1112189.9031 -4842955.0319  3985352.2376                  APPROX POSITION XYZ

rinexpvt -o usno0200.05o -n brdc0200.05n -m usno0200.05m -e "1112189.9031 -4842955.0319  3985352.2376" 

2005 1 20 00 00 0.000000 -2.64323278089 1.39273684601 -4.44579107837
2005 1 20 00 00 30.000000 0.675237880363 -0.413842977087 -1.86595014292
2005 1 20 00 01 0.000000 0.74075583724 -1.74562915823 4.70500528956
2005 1 20 00 01 30.000000 -0.514261859844 -0.427112579929 3.92260789052
2005 1 20 00 02 0.000000 -0.798000760765 0.225433490228 1.2040383517
2005 1 20 00 02 30.000000 -0.905046895282 -0.156961158353 -0.0422947010004
2005 1 20 00 03 0.000000 -0.144614169541 1.78733342562 -2.7520269223
2005 1 20 00 03 30.000000 0.432346745093 -0.229710319547 -0.872438081029
2005 1 20 00 04 0.000000 -1.19640547889 -1.09032516564 2.40032451942
2005 1 20 00 04 30.000000 0.670398976551 1.68417470062 -4.01206558988
\end{verbatim}
\end{singlespace}
\normalsize


\subsection{Generating Positions using Precise Ephemerides}

Precise ephemerides may be substituted for broadcast ephemerides. The precise
ephemerides must be in the SP-3 file format. In order to process a given
period of RINEX observations, precise ephemerides must be utilized for times
before and after that period in order to eliminate interpolation effects. For
this reason the \texttt{-p} option to specify a precise ephemeris can be
repeated. The following example demonstrates the use of precise ephemerides.

\small
\begin{singlespace}
\begin{verbatim}
rinexpvt -o usno0200.05o -m usno0200.05m -e \"1112189.9031 -4842955.0319
3985352.2376\" -p nga13063.apc -p nga13064.apc -p nga13065.apc 

2005 1 20 00 00 0.000000 -2.08992486501 1.51632100425 -5.38603364386
2005 1 20 00 00 30.000000 1.22849620836 -0.281974793322 -2.78857829356
2005 1 20 00 01 0.000000 1.29400421219 -1.60589389988 3.7997452854
2005 1 20 00 01 30.000000 0.03901809145 -0.279940424961 3.0344477269
2005 1 20 00 02 0.000000 -0.141549865351 0.430550617004 0.613916181074
2005 1 20 00 02 30.000000 -0.247842106971 0.0554617639326 -0.61408409222
2005 1 20 00 03 0.000000 0.513345520633 2.00662083505 -3.30594517769
2005 1 20 00 03 30.000000 1.18397434384 0.173306575113 -1.80221181762
2005 1 20 00 04 0.000000 -0.445369991941 -0.681710302654 1.48775750121
2005 1 20 00 04 30.000000 1.26854804266 1.86268110989 -4.31594225054
\end{verbatim}
\end{singlespace}
\normalsize


\subsection{Emulating Standard Positioning Service (SPS) Performance}

By default, \texttt{rinexpvt} will attempt to form the best possible position
for each epoch of observation. If dual frequency observations are applied,
they are used. If for a single epoch, only C/A observations are available,
then it is used. In order to perform solutions using only C/A, the \texttt{-s}
switch is available. The following is a demonstration of this switch.

\small
\begin{singlespace}
\begin{verbatim}
rinexpvt -o usno0200.05o -n brdc0200.05n -m usno0200.05m -s

2005 1 20 00 00 0.000000 1112192.36858 -4842952.68698 3985350.17084
2005 1 20 00 00 30.000000 1112190.34546 -4842953.75694 3985351.57171
2005 1 20 00 01 0.000000 1112191.29632 -4842954.16477 3985353.65599
2005 1 20 00 01 30.000000 1112191.97305 -4842954.21052 3985353.96079
2005 1 20 00 02 0.000000 1112191.47444 -4842954.60185 3985351.44099
2005 1 20 00 02 30.000000 1112191.67217 -4842953.79149 3985352.74304
2005 1 20 00 03 0.000000 1112192.35285 -4842953.76184 3985351.25908
2005 1 20 00 03 30.000000 1112189.43589 -4842951.88681 3985348.73888
2005 1 20 00 04 0.000000 1112190.55705 -4842953.10278 3985349.95615
2005 1 20 00 04 30.000000 1112188.71119 -4842952.128 3985348.15393
\end{verbatim}
\end{singlespace}
\normalsize


\subsection{Emulating Precise Positioning Service (PPS) Performance}

Similar to the option described in the previous subsection, there is an option
to limit the solutions of \texttt{rinexpvt} to those strictly derived from
dual frequency observations. The following session demonstrates this switch.

\small
\begin{singlespace}
\begin{verbatim}
rinexpvt -o usno0200.05o -n brdc0200.05n -m usno0200.05m -d

2005 1 20 00 00 0.000000 1112192.67926 -4842951.98205 3985348.06329
2005 1 20 00 00 30.000000 1112188.65938 -4842953.48346 3985351.48398
2005 1 20 00 01 0.000000 1112189.48576 -4842957.45711 3985356.92698
2005 1 20 00 01 30.000000 1112191.15384 -4842957.53284 3985355.11895
2005 1 20 00 02 0.000000 1112191.1508 -4842955.89459 3985352.76549
2005 1 20 00 02 30.000000 1112190.99828 -4842954.61737 3985352.3681
2005 1 20 00 03 0.000000 1112189.92412 -4842954.29518 3985348.72842
2005 1 20 00 03 30.000000 1112189.16937 -4842954.31307 3985351.92146
2005 1 20 00 04 0.000000 1112191.62124 -4842955.6613 3985354.82972
2005 1 20 00 04 30.000000 1112188.66733 -4842953.49038 3985348.04322}}
\end{verbatim}
\end{singlespace}
\normalsize


\subsection{Customizing the Epoch Format}

The GPSTk library supports conversion among a number of time formats. This
conversion ability is provided to the end user of \texttt{rinexpvt} in the
form of the \texttt{-t} command line switch and its argument, a string
describing the time format. The GPSTk documentation to DayTime's printf method
contains a full list of specifiers that can be used within the time format
string. The following table summarizes some of these options.



\begin{center}
  \begin{tabular}{ll}
    \% Y & Four digit year\\
    \% y & Year modulo 100\\
    \% m & Month number\\
    \% b & Month name\\
    \% d & Day of month\\
    \% S & Second of minute\\
    \% F & Full GPS week\\
    \% Z & Z count\\
    \% g & Seconds of week\\
    \% j & Day of year\\
    \% s & Seconds of day\\
    \% Q & Modified Julian Date
  \end{tabular}
\end{center}

\begin{center}
  Table. Time Format Specifiers
\end{center}



In the following example we see how to apply the format specifiers in the form
of a string.

\small
\begin{singlespace}
\begin{verbatim}
rinexpvt -o usno0200.05o -n brdc0200.05n -m usno0200.05m -t \"%F %g\" 
1306 345600.000000 1112192.67926 -4842951.98205 3985348.06329
1306 345630.000000 1112188.65938 -4842953.48346 3985351.48398
1306 345660.000000 1112189.48576 -4842957.45711 3985356.92698
1306 345690.000000 1112191.15384 -4842957.53284 3985355.11895
1306 345720.000000 1112191.1508 -4842955.89459 3985352.76549
1306 345750.000000 1112190.99828 -4842954.61737 3985352.3681
1306 345780.000000 1112189.92412 -4842954.29518 3985348.72842
1306 345810.000000 1112189.16937 -4842954.31307 3985351.92146
1306 345840.000000 1112191.62124 -4842955.6613 3985354.82972
1306 345870.000000 1112188.66733 -4842953.49038 3985348.04322
\end{verbatim}
\end{singlespace}
\normalsize


\section{References}

\begin{enumerate}
  \item The GPS Toolkit, GPSTk. Website: http://www.gpstk.org/.
  
  \item RINEX: The Receiver Independent Exchange Format Version 2.10.
  Available on the web at http://www.ngs.noaa.gov/CORS/Rinex2.html.
  
  \item Hofmann-Wellenhoff, B., Lichtenegger, H., and Collins, J.
  \textit{Global Positioning Theory:} \textit{Theory and Practice}, fifth ed.
  Springer-Verlag. 2004.
  
  \item Parkinson, Bradford W. and Spilker, James J., editors. Global
  Positioning Theory: Theory and Applications, Volume I. AIAA Press, 1996.
  
  \item Borre, Kai and Strang, Gilbert. \textit{Linear Algebra, Geodesy and
  GPS}. Wellesley-Cambridge Press, 1997.
  
  \item The GPS Interface Control Document (ICD-GPS-200), which can be found
  at http://www.navcen.uscg.gov/ftp/policy/icd200/ICD200Cw1234.pdf.
  
  \item Goad, C. C. and Goodman, L. ``A modified tropospheric refraction
  correction model.'' \textit{Proceeding of the Annual American Geophysical
  Union Fall Meeting}, San Francisco, 1974.
\end{enumerate}

