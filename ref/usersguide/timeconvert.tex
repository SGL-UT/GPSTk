%\documentclass{article}
%\usepackage{fancyvrb}
%\usepackage{src/perltex}
%\usepackage{src/xcolor}
%\usepackage{listings}
%\usepackage{longtable}
%\usepackage{multirow}
%\RecustomVerbatimEnvironment{Verbatim}{Verbatim}{frame=single}
\definecolor{console}{rgb}{0.95,0.95,0.95}
\lstset{basicstyle=\ttfamily, columns=flexible, backgroundcolor=\color{console}}

\newcommand{\outputsize}{footnotesize}
\newcommand{\application}[1]{\emph{#1}}
\newcommand{\setconsole}{\lstset{basicstyle=\ttfamily, columns=flexible, backgroundcolor=\color{console}}}
\newcommand{\setfileio}{\lstset{basicstyle=\ttfamily, columns=flexible, backgroundcolor=\color{console}}}

\perlnewcommand{\getuse}[1]
{
        my $command = $_[0];
        $command = $command." > temp 2>&1 |head -n 15";
        system("$command");

	my $counter = 0;
	my $done = 0;
	my $output = "";
        open(input,"temp");

        while(my $line = <input>)
	{
		if($done == 0)
		{
			$output = $output.$line if $counter < 15;
			$output = $output." . . .\n" if $counter >= 15;
			$done = 1 if $counter >= 15;
		}

		$counter = $counter + 1;
	}

        close(input);

        return  "\\begin{\\outputsize}\n" . "\\begin{lstlisting}\n" .
                "> ".$_[0]."\n\n" . $output .
                "\\end{lstlisting}\n" . "\\end{\\outputsize}\n";
}

\perlnewcommand{\getrevision}[1]
{
	my $revision = $_[0];
	$revision =~ m/\$LastChangedRevision: ([^\$]*)/;
	return $1;
}

\perlnewcommand{\entry}[4]
{
	my $output = "$_[0] \& $_[1] \& \\multirow{$_[3]}{2.5in}{$_[2]} \\\\";

	my $cnt = 0;
	while($cnt < ($_[3]-1))
	{
		$output = $output." \& \& \\\\ ";
		$cnt = $cnt + 1;
	}
	
	return $output;
}


%\begin{document}

\index{timeconvert!application writeup}
\section{\emph{timeconvert}}
\subsection{Overview}
This application allows the user to convert among time formats associated with 
GPS. Time formats include: civilian time, Julian day of year and year, GPS week 
and seconds of week, Z counts, and Modified Julian Date (MJD).

\subsection{Usage}
\begin{\outputsize}
\begin{longtable}{lll}
\multicolumn{3}{l}{\textbf{Optional Arguments}} \\
\entry{Short Arg.}{Long Arg.}{Description}{1}
\entry{-d}{--debug}{Increase debug level}{1}
\entry{-v}{--verbose}{Increase verbosity}{1}
\entry{-h}{--help}{Print help usage}{1}
\entry{-c}{--calendar=TIME}{``Month(numeric) DayOfMonth Year"}{1}
\entry{-r}{--rinex=TIME}{"Month(numeric) DayOfMonth Year Hour:Minute:Second"}{2}
\entry{-R}{--rinex-file=TIME}{"Year(2-digit) Month(numeric) DayOfMonth Hour Minute Second"}{2}
\entry{-y}{--doy=TIME}{"Year DayOfYear SecondsOfDay"}{1}
\entry{-m}{--mjd=TIME}{"ModifiedJulianDate"}{1}
\entry{-o}{--shortweekandsow=TIME}{"10bitGPSweek SecondsOfWeek Year"}{1}
\entry{-z}{--shortweekandzcounts=TIME}{"10bitGPSweek ZCounts Year"}{1}
\entry{-f}{--fullweekandsow=TIME}{"FullGPSweek SecondsOfWeek"}{1}
\entry{-w}{--fullweekandzcounts=TIMEo}{"FullGPSweek ZCounts"}{1}
\entry{-u}{--unixtime=TIME}{"UnixSeconds UnixMicroseconds"}{1}
\entry{-Z}{--fullZcounts=TIME}{"fullZcounts"}{1}
\entry{-F}{--format=ARG}{Time format to use on output}{1}
\entry{-a}{--add-offset=NUM}{add NUM seconds to specified time}{1}
\entry{-s}{--sub-offset=NUM}{subtract NUM seconds from specified time}{1}
\end{longtable}
\end{\outputsize}

\subsection{Examples}
\getuse{timeconvert -r "05 06 1985 13:50:02"}

\getuse{timeconvert -o "1379 500 2006"}

\getuse{timeconvert -o "1379 500 2006 -a 86400"}

\getuse{timeconvert -w "1381 500" -s 200}

\subsection{Notes}

%\end{document}

