%\documentclass{article}
%\usepackage{fancyvrb}
%\usepackage{perltex}
%\usepackage{xcolor}
%\usepackage{listings}
%\usepackage{longtable}
%\usepackage{multirow}
%\RecustomVerbatimEnvironment{Verbatim}{Verbatim}{frame=single}
\definecolor{console}{rgb}{0.95,0.95,0.95}
\lstset{basicstyle=\ttfamily, columns=flexible, backgroundcolor=\color{console}}

\newcommand{\outputsize}{footnotesize}
\newcommand{\application}[1]{\emph{#1}}
\newcommand{\setconsole}{\lstset{basicstyle=\ttfamily, columns=flexible, backgroundcolor=\color{console}}}
\newcommand{\setfileio}{\lstset{basicstyle=\ttfamily, columns=flexible, backgroundcolor=\color{console}}}

\perlnewcommand{\getuse}[1]
{
        my $command = $_[0];
        $command = $command." > temp 2>&1 |head -n 15";
        system("$command");

	my $counter = 0;
	my $done = 0;
	my $output = "";
        open(input,"temp");

        while(my $line = <input>)
	{
		if($done == 0)
		{
			$output = $output.$line if $counter < 15;
			$output = $output." . . .\n" if $counter >= 15;
			$done = 1 if $counter >= 15;
		}

		$counter = $counter + 1;
	}

        close(input);

        return  "\\begin{\\outputsize}\n" . "\\begin{lstlisting}\n" .
                "> ".$_[0]."\n\n" . $output .
                "\\end{lstlisting}\n" . "\\end{\\outputsize}\n";
}

\perlnewcommand{\getrevision}[1]
{
	my $revision = $_[0];
	$revision =~ m/\$LastChangedRevision: ([^\$]*)/;
	return $1;
}

\perlnewcommand{\entry}[4]
{
	my $output = "$_[0] \& $_[1] \& \\multirow{$_[3]}{2.5in}{$_[2]} \\\\";

	my $cnt = 0;
	while($cnt < ($_[3]-1))
	{
		$output = $output." \& \& \\\\ ";
		$cnt = $cnt + 1;
	}
	
	return $output;
}


%\begin{document}

\index{daa!application writeup}
\section{\emph{daa}}
\subsection{Overview}
This application performs a data availability analysis of the input data. In general, availability is determined by station and satellite position.

\subsection{Usage}
\begin{\outputsize}
\begin{longtable}{lll}
\multicolumn{3}{c}{\application{daa}} \\
\multicolumn{3}{l}{\textbf{Required Arguments}} \\
\entry{Short Arg.}{Long Arg.}{Description}{1}
\entry{-e}{--eph=ARG}{Where to get the ephemeris data.  Acceptable formats include RINEX nav, FIC, MDP, SP3, YUMA, and SEM.  Repeat for multiple files.}{3}
\entry{-o}{--obs=ARG}{Where to get the observation data.  Acceptable formats include RINEX obs, MDP, smooth, Novatel, and raw Ashtech.  Repeat for multiple files.  If a RINEX obs file is provided, the position will be taken from the header unless otherwise specified.}{5}
& & \\
\multicolumn{3}{l}{\textbf{Optional Arguments}} \\
\entry{Short Arg.}{Long Arg.}{Description}{1}
\entry{-d}{--debug}{Increase debug level.}{1}
\entry{-v}{--verbose}{Increase verbosity.}{1}
\entry{-h}{--help}{Print help usage.}{1}
\entry{}{--ouput=ARG}{Output location (default is stdout).}{1}
\entry{-x}{--independent=ARG}{The independent variable in the analysis.  The default is time.}{2}
\entry{-c}{--msc=ARG}{Station coordinates file.}{1}
\entry{-m}{--msid=ARG}{Station for which to process data.  Used to select a station position from the msc file.}{2}
\entry{-t}{--time-format=ARG}{CommonTime format specifier used for times in the output.  The default is \lq\lq\%Y \%j \%02H:\%02M:\%04.1f\rq\rq.}{3}
\entry{}{--mask-angle=ARG}{Ignore anomalies on SVs below this elevation.  
	The default is 10 degrees.}{2}
\entry{}{--track-angle=ARG}{Assume the receiver starts tracking at this elevation.
	The default is 10 degrees.}{2}
\entry{}{--time-mask=ARG}{Ignore anomalies on SVs that haven't been above the mask angle for this number of seconds.
	The default is 0 seconds.}{3}
\entry{}{--snr=ARG}{Discard data with an SNR less than this value. 
	The default is 20 dB-Hz.}{2}
\entry{-p}{--position=ARG}{Receiver antenna position in Position (x,y,z) coordinates. 
	Format as a string: "X Y Z".}{2}
\entry{-l}{--time-span=ARG}{How much data to process, in seconds}{1}
\entry{}{--ignore-prn=ARG}{Specify the PRN of an SV to not report on in the output. Repeat to specify multiple SVs.}{2}
\entry{}{--obs-interval=ARG}{Specify the time interval, in seconds, between observations. The default is to scan the file to discover this via examination of the file.}{3}
\entry{-b}{--bad-health}{Ignore anomalies associated with SVs that are marked unhealthy.}{2}
\entry{-s}{--smash-adjacent}{Combine adjacent lines from the same PRN.}{1}
\entry{}{--start-time=TIME}{Ignore data before this time.
	\%4Y/\%03j/\%02H:\%02M:\%05.2f}{2}
\entry{}{--stop-time=TIME}{Ignore any data after this time.}{1}
\end{longtable}
\end{\outputsize}
%\end{document}
