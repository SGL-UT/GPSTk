%\documentclass{article}
%\usepackage{fancyvrb}
%\usepackage{perltex}
%\usepackage{xcolor}
%\usepackage{listings}
%\usepackage{longtable}
%\usepackage{multirow}
%\RecustomVerbatimEnvironment{Verbatim}{Verbatim}{frame=single}
\definecolor{console}{rgb}{0.95,0.95,0.95}
\lstset{basicstyle=\ttfamily, columns=flexible, backgroundcolor=\color{console}}

\newcommand{\outputsize}{footnotesize}
\newcommand{\application}[1]{\emph{#1}}
\newcommand{\setconsole}{\lstset{basicstyle=\ttfamily, columns=flexible, backgroundcolor=\color{console}}}
\newcommand{\setfileio}{\lstset{basicstyle=\ttfamily, columns=flexible, backgroundcolor=\color{console}}}

\perlnewcommand{\getuse}[1]
{
        my $command = $_[0];
        $command = $command." > temp 2>&1 |head -n 15";
        system("$command");

	my $counter = 0;
	my $done = 0;
	my $output = "";
        open(input,"temp");

        while(my $line = <input>)
	{
		if($done == 0)
		{
			$output = $output.$line if $counter < 15;
			$output = $output." . . .\n" if $counter >= 15;
			$done = 1 if $counter >= 15;
		}

		$counter = $counter + 1;
	}

        close(input);

        return  "\\begin{\\outputsize}\n" . "\\begin{lstlisting}\n" .
                "> ".$_[0]."\n\n" . $output .
                "\\end{lstlisting}\n" . "\\end{\\outputsize}\n";
}

\perlnewcommand{\getrevision}[1]
{
	my $revision = $_[0];
	$revision =~ m/\$LastChangedRevision: ([^\$]*)/;
	return $1;
}

\perlnewcommand{\entry}[4]
{
	my $output = "$_[0] \& $_[1] \& \\multirow{$_[3]}{2.5in}{$_[2]} \\\\";

	my $cnt = 0;
	while($cnt < ($_[3]-1))
	{
		$output = $output." \& \& \\\\ ";
		$cnt = $cnt + 1;
	}
	
	return $output;
}


%\begin{document}

\index{ash2mdp!application writeup}
\index{ash2xyz!application writeup}
\section{\emph{ash2mdp ash2xyz}}
\subsection{Overview}
These applications process Ashtech Z(Y)-12 observation and ephemeris data and output satellite positions and ionospheric corrections in either MDP or XYZ format.
\subsection{Usage}

\subsubsection{\emph{ash2mdp ash2xyz}}

\begin{\outputsize}
\begin{longtable}{lll}
%\multicolumn{3}{c}{\application{ash2mdp} \application{ash2xyz}} \\
\multicolumn{3}{l}{\textbf{Optional Arguments}} \\
\entry{Short Arg.}{Long Arg.}{Description}{1}
\entry{-i}{}{Where to get data from.  The default is to use stdin.}{2}
\entry{-o}{}{Where to send the output.  The default is to use stdout.}{2}
\entry{-d}{--debug}{Increase debug level.}{1}
\entry{-v}{--verbose}{Increase verbosity.}{1}
\entry{-h}{--help}{Print help usage.}{1}
\entry{-w}{--week=NUM}{The full GPS week in which this data starts.  Use this option when the start time of the data being processed is not during this week.}{3}
\entry{-c}{--code=ARG}{Restriction for source of observation data collected via L1/L2 Y code tracking will be used.  Options are "Y", "P", and "codeless." XYZ only.}{4}
\entry{-s}{--offset=NUM}{Output SV positions at a time offset from the current time.  Give a positive or negative integer of seconds. XYZ only.}{3}
\entry{-n}{--num\_points=NUM}{Width of the exponential filter moving window, in number of points (default is 36).  XYZ only.}{3}
\end{longtable}
\end{\outputsize}

\subsection{Notes}
Input is on the command line, or of the same format in a file (-f$<$file$>$).

%\end{document}
